\documentclass[11pt,a4paper]{moderncv}
\moderncvtheme[blue]{classic}
\usepackage[utf8]{inputenc}
\usepackage[top=2cm, bottom=1.1cm, left=2cm, right=2cm]{geometry}
% Largeur de la colonne pour les dates
\setlength{\hintscolumnwidth}{3.5cm}

\photo[64pt][0.pt]{photo.pdf}
\firstname{Félix}
\familyname{Piédallu}
\title{Étudiant en Physique \& Nanosciences\newline à Phelma, INP}
\address{15 rue Buffon}{38000 Grenoble}
\email{felix@piedallu.me}
\homepage{geekolloc.fr}
\mobile{06 51 41 32 48}
\extrainfo{20 ans}

\begin{document}
\maketitle
%\cventry{Objet :}{Stage opérateur de Première Année}{}{}{}{}
\vspace*{-5mm}
\section{Formation}
\cventry{2015 -- 2016}{Troisième année d'école d'Ingénieur}{Phelma}{Grenoble INP}{}{Filière Physique \& NanoSciences, Spécialité Optique-Microélectronique}
\cventry{2014 -- 2015}{Deuxième année d'école d'Ingénieur}{Phelma}{Grenoble INP}{}{Filière Physique \& NanoSciences}
\cventry{2013 -- 2014}{Première année d'école d'Ingénieur}{Phelma}{Grenoble INP}{}{Section Physique, Matériaux, Procédés (Cours d'Ouverture au Magnétisme)}
\cventry{2011 -- 2013}{Classe Préparatoire}{Lycée Charlemagne}{Paris}{Maths-Physique (MP*)}{}
\cventry{2011}{Baccalauréat Scientifique}{Lycée Claude Monet}{Paris}{}{}


\section{Réalisations techniques}
\cventry{2016}{Stage de Fin d'Études}{Institut Néel}{Nano-Optiques \& Forces}{Grenoble}{Développement de nano-pinces fibrées plasmoniques}
\cventry{2015}{Stage d'Application}{LPA}{Équipe Hybrid Quantum Circuits}{Paris}{Mise en place d'une expérience à très basse température et étude d'effets quantiques dans des systèmes nanométriques}
\cventry{2014 -- 2015}{Président du Club Robotronik de Phelma}{}{}{}{}
\cventry{2014}{Stage Opérateur}{}{à la DSI du Courier de La Poste}{}{Intégration sous Android d'un service de reconnaissance d'images sur serveur distant.}
\cventry{2013 -- 2014}{Participation à la Coupe de France de Robotique}{}{}{}{}
\cventry{Maths Spé}{Travaux d'Intérêt Personnel Encadré}{}{avec Guillaume Poly}{}{Application du Théorème de Superposition de Kolmogorov à la compression d'images.}


\section{Connaissances en informatique}
\cvitem{Langages maîtrisés}{C, C++, GTK+, \LaTeX, Java (Android), Bash (Linux/UNIX)}
\cvitem{Autres langages}{HTML, Python}
\cvitem{Systèmes}{Linux (Arch, Debian \& Ubuntu), Raspberry Pi}


\section{Compétences Linguistiques}
\cvlanguage{Allemand}{Lu, écrit, parlé}{Première langue entre 2004 et 2013, séjours en Allemagne}
\cvlanguage{Anglais}{Lu, écrit, parlé couramment}{Niveau C2}
\cvlanguage{Chinois Mandarin}{Lu, écrit, parlé}{10 ans d'apprentissage et un séjour en Chine}
\cvlanguage{Russe}{Bases acquises}{}


\section{Centres d'intérêt}
\cvitem{Informatique}{Participe activement à la rédaction du Wiki Ubuntu francophone}
\cvitem{}            {Développement d'une simulation GTK+ pour le club Robotronik}
\cvitem{Musique}{Pratique l'Accordéon depuis 9 ans, écoute du Jazz au Metal}
\cvitem{Plongée, Natation}{Diplôme de niveau 1}
\cvitem{Tir à l'arc}{Première année}
\cvitem{Modélisme}{Aéromodélisme et Robotique}

\end{document}
